\section{Introduction}

\textbf{Put your paper here.}
\vskip 0.4in

This is the Rubin Observatory overview paper: \citet{2019ApJ...873..111I}.

\begin{itemize}
    \item Explain Astrometry.
    \item Requirment for Rubin.
    \item Important for Lensing, solar system physics.
    \item DES observed E-mode and static component.
    \item Similar result were observed in HSC.
    \item Early result of LSSTComCam are showing this to. 
    \item Goal here is to characterize for Rubin to see how perform the global astrometric solution.
    \item See how Turbulence is behaving on Rubin with shorter exposure time and see any static effect.
    \item Use data from DP2 to assess this from LSSTCam took in the first few month of the camera beeing on sky.
\end{itemize}


\section{Astrometric solution and residuals from LSSTCam}

\subsection{Astrometric solution from LSSTCam}

\begin{itemize}
    \item Rubin software is used to process the image.
    \item Astrometry is compute using a modified version of DES Astrometry.
    \item Solution is compute using all visit from a given tract.
    \item Match between detected source and Gaia.
    \item A polynomial for fov and a polynomial per detector.
    \item fit proper motion.
    \item Residuals are compute to be the diff between mean object position and position in current visit.
    \item We look at residual in Focal plane coordinate system.
    \item x and y are position into focal plane and dx and dy astrometric residuals.
\end{itemize}

\subsection{Characterization of Astrometric residuals of LSSTCam}

\begin{itemize}
    \item Show residuals in some visit at different time. 
    \item Show E/B mode distribution.
    \item Show that some visit exibit B-mode.
    \item Compare with PSF ?
    \item E mode vs time ?
    \item Discuss that way stronger than DES, and HSC. Make a comment compared to DP1.
\end{itemize}

\section{Gaussian Process modeling of Astrometric residuals at single visit level}


\subsection{The GP model}

\begin{itemize}
    \item This is exactly the same thing as in HSC
    \item Just remind quickly
    \item Explain what is different with DES again
    \item Say O(n * log(n))
\end{itemize}

\subsection{Practical implementation for Rubin}

\begin{itemize}
    \item Randomly select 10k source.
    \item Interpolate CCD per CCD using the full focal plane kernel.
    \item Does not look to affect performance.
\end{itemize}

\subsection{Results}

\begin{itemize}
    \item Show some visit result.
    \item Explain that with dense field looks GP, still some remaining stuff. 
    \item Final correlation.
    \item Make same plot than DES to have side to side comparaison? 
\end{itemize}

\section{Static astrometric distortion}


\subsection{Full Focal plane vue}

\begin{itemize}
    \item Show full FoV distorsion
    \item Show some CCD with some filtering
    \item Show what happened when closing the loop
\end{itemize}

\subsection{On ITL distortion}

\begin{itemize}
    \item Average plot and comment it looks like SLAC measurement. 
\end{itemize}

\subsection{On E2V distorsion}

\begin{itemize}
    \item Average plot and comment not fully understood as I am writting those line but looks like the SLAC measurment to. 
\end{itemize}



